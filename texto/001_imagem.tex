{\tiny {\tiny \documentclass[a4paper,12pt]{article}

% Fonte e matemática elegante
\usepackage{mathpazo}
\usepackage{bm}

% Idioma e codificação
\usepackage[utf8]{inputenc}
\usepackage[T1]{fontenc}
\usepackage[brazil]{babel}

% TikZ para desenhar
\usepackage{tikz}
\usetikzlibrary{calc}

% Cores personalizadas
\definecolor{tcolor}{RGB}{255,127,0}
\definecolor{lcolor}{RGB}{255,178,102}
\definecolor{pcolor}{RGB}{251,204,231}

% Margem lateral suficiente para as caixas
\setlength{\marginparwidth}{2.5cm}

% Macro principal para anotações elegantes
\newcommand{\elegantpar}[2]{%
	\textcolor{tcolor}{$\bm\langle{}\!{}$#1${}\!{}\bm\rangle$}%
	\begin{tikzpicture}[remember picture, baseline=-0.75ex]%
		\node[coordinate] (inText) {};%
	\end{tikzpicture}%
	\marginpar{%
		\renewcommand{\baselinestretch}{1.0}%
		\begin{tikzpicture}[remember picture]%
			\draw node[fill=pcolor, rounded corners, text width=\marginparwidth] (inNote) {\footnotesize #2};%
		\end{tikzpicture}%
	}%
	\begin{tikzpicture}[remember picture, overlay]%
		\draw[draw = lcolor, thick]
		([yshift=-0.55em] inText)
		-| ([xshift=-0.55em] inNote.west)
		-| (inNote.west);%
	\end{tikzpicture}%
}

\begin{document}
	
	\section*{Exemplo com Anotações Laterais}
	
	A leitura crítica é uma competência essencial no século XXI, especialmente no contexto da educação mediada por tecnologias. Compreender o papel da tecnologia na aprendizagem exige não apenas habilidades técnicas, mas também uma \elegantpar{reflexão ética e pedagógica}{As anotações laterais ajudam a expandir o conteúdo sem interromper o fluxo do texto principal}, promovendo decisões mais conscientes.
	
	\vspace{1em}
	
	O papel do(a) professor(a) está em constante transformação. Hoje, espera-se que ele(a) atue como mediador(a), facilitador(a) da aprendizagem, e não apenas como transmissor(a) de conteúdo. Essa mudança \elegantpar{exige novas competências docentes}{Como a capacidade de desenhar experiências de aprendizagem significativas, usar recursos digitais com intencionalidade e dialogar com a diversidade cultural dos estudantes}.
	
	\vspace{1em}
	
	Criar documentos tipograficamente bem construídos com \LaTeX\ também é uma forma de comunicar conhecimento com clareza e estética. A macro \texttt{\textbackslash elegantpar} é um exemplo de \elegantpar{personalização inteligente}{Você pode usar esse tipo de anotação em artigos científicos, materiais didáticos ou até portfólios profissionais}, valorizando o design da informação.
	
\end{document}}}
